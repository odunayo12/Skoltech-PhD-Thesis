\chapter{Discussion}
\label{cap:discussion}
% \epigraphhead[50]{%
%     \epigraph{"if I have seen further it is by standing on the shoulders of Giants."}{Isaac Newton, 1675}
% }
\todo[inline]{TODO: Review by Mr Isaiah Oluwabunmi Adebayo}
\subsection{Problem Formulation}
What if in a graph of connected components $G[\tilde{S}]$, the maximum number of nodes in a connected components are equal? This may not necessarily imply that the cost of disconnectivity (pairwise connectivity as a measure, say) is the same for all disconnected components. Therefore, might not solve the problem of disconnectivity in a network breakdown.
\begin{figure}[htb!]
    \centering \includegraphics[width=\textwidth]{graphics/conn_comp.png}
    \caption{Components}
    \label{fig:conn_comp}
\end{figure}
For example, in the Graph above ${c_1, c_3, c_4}$ (in red in \ref{fig:conn_comp}) all have same value of maximum number of connected nodes which may not have same cost should there be weights - representing capacities or resilience index - attached to each node.
\subsection{Possible way out}
What if we formulate our objectives in such a way that:
\begin{enumerate}
    \item It first seek the maximum number of connected nodes in the subgraph; then \label{obj_1}
    \item Picks the one with the relevant figure of pairwise connectivity? \label{obj_2}
\end{enumerate}
By \ref{obj_1}, we assume the authors default structure. By \ref{obj_2}, we are able to extend their work.
If we succeed, the novelty will be the consideration of a scheme that considers both problems simultaneously through a sequential algorithm that counts first and evaluates next. That is, we will be combining $CNDP_a^1$ and $CNDP_b^2$; see \ref{fig:cdnp} below as in \cite{rezaei2020eia}.
\begin{figure}[htb!]
    \centering \includegraphics[width=\textwidth]{graphics/cndp.png}
    \caption{CND Problem Types}
    \label{fig:cdnp}
\end{figure}

\subsection{Challenges}
Can't say for now how the simulation will look like. I mean I cannot predict the feasibility. But if we agree on this objective, the formulation is quite "theoretically or mathematically" feasible and explainable. But theory do not always guarantee feasibility in programming. Let us continue from here.
% We inspired our work from \citet{Chakrabarti2014}.
% \acrshort{sysml} is the reference in this field \citep{ObjectManagementGroup2015}
% The tools keep evolving \citep{Skoltech2017}.

\subsection{test}
we are doing well.